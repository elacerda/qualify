%%%%%%%%%%%%%%%%%%%%%%%%%%%%%%%%%%%%%%%%%%%%%%%%%%%%%%%%%%%%%%%%%
% Qualificacao de Doutorado / Dept Fisica, CFM, UFSC            %
% Eduardo@UFSC - 2015                                           %
%%%%%%%%%%%%%%%%%%%%%%%%%%%%%%%%%%%%%%%%%%%%%%%%%%%%%%%%%%%%%%%%%

%:::::::::::::::::::::::::::::::::::::::::::::::::::::::::::::::%
%                                                               %
%                          Capítulo 3                           %
%                                                               %
%:::::::::::::::::::::::::::::::::::::::::::::::::::::::::::::::%

%***************************************************************%
%                                                               %
%                     Comparação SFR EML                        %
%                                                               %
%***************************************************************%

\chapter{Propriedades da síntese e propriedades nebulares}
\label{sec:synvsneb}

\section{Amostra de galáxias}
\label{sec:synvsneb:amostra}

Neste trabalho estamos interessados em estudar a formação estelar recente em discos de galáxias.
Nossa amostra começa conténdo todas as 226176 regiões (zonas) das 305 galáxias espirais do CALIFA
{\em Survey}. Cada uma dessas zonas dessa é composta por um ou mais píxels, com um espectro
resultante da soma dos espectros destes, para que tenhamos relação sinal-ruído maior ou igual a 20
na janela de normalização do espectro resultante. Esse procedimento, conhecido como {\em Voronoi
binning}, está detalhado, juntamente com o procedimento de derivação das propriedades estelares
através do código \starlight para cada uma das regiões destas galáxias, em
\citet{CidFernandes.etal.2013a}.

\subsection{Classificação Morfológica}
\label{sec:synvsneb:amostra:morf}

Com tipos morfológicos variando entre Sa e Sd, massas estelares entre $10^9$ e $10^{11.5}$ massas
solares e populações estelares com idades médias entre $10^8$ e $10^{10}$ anos, podemos ver na Fig.
\ref{fig:amostraMorf} que as galáxias se ordenam de forma interessante quando agrupadas por tipo
morfológico, anticorrelacionando com a idade média estelar e a massa estelar ($M_\star$ e $t_\star$)
e correlacionando com a fração de luz proveniente das população jovens ($x_Y \equiv x_Y(t_\star <
t_{SF})$). Cada galáxia contribui com um ponto em cada painel deste gráfico, ou seja, são
propriedades integradas. Os intervalos entre primeiro e terceiro quartil quase não se sobrepõem no
quando analisamos as classes morfológicas por idade média. O resultado parece ser muito interessante
visto que a classificação morfológica foi feita por colaboradores do CALIFA totalmente através de
inspeção visual das imagens na banda-r do \SDSS das mesmas galáxias. Vemos também que as galáxias
tipo Sd possuem as populações estelares mais jovens e menos massivas na média. Por ser um fenômeno
apenas de posição do referencial de observação não deveríamos ver preferência por valor de relação
de semi-exios (b/a) quando dividimos em classes morfológicas, o que realmente acontece.

\begin{figure}
	\centering
	%\resizebox{0.99\textwidth}{!}{\includegraphics{Nh_logt_metBase_Padova2000_salp.pdf}}
	\resizebox{0.99\textwidth}{!}{\includegraphics{figuras/sample_morf.pdf}}
	\caption[Classificação por morfologia.]
	{\emph{Em sentido horário a partir do painel superior esquerdo}: gráfico da massa estelar, idade
	estelar média, fração de luz proveniente das população jovens ($x_Y \equiv x_Y(t_\star < t_{SF})$) e
	inclinação, divididos em classes morfológicas para 305 galáxias espirais da amostra total do
	CALIFA. No primeiro painel, temos o número de galáxias dentro de cada classe morfológica. Cada
	caixa tem altura definida pelo primeiro e terceiro quartil da distribuição dentro de um tipo
	morfológico. Uma faixa preta marca a mediana e uma estrela a média. Em cada caixa, a linha
	pontilhada vertical se estende mostrando o intervalo de $3\sigma$. Valores que ficam fora do
	intervalo de $3\sigma$ são marcados por uma cruz vermelha.}
	\label{fig:amostraMorf}
\end{figure}

\subsection{Mascarando elementos e removendo {\em outliers}}
\label{sec:synvsneb:amostra:mask}

Para que possamos focar nossos estudos às regiões de formação estelar destas galáxias, aplicamos uma
máscara nos dados selecionando as regiões que possuam:
\begin{itemize}
  \item medidas do fluxo integrado das linhas de \Halpha, \Hbeta, \nII e \oIII com relação
sinal-ruído maior do que 3;
  \item medidas para as seis propriedades comparadas neste capítulo ($\tauVS$, $\tauVN$,
$\SigmaSFR$, $\SigmaSFRN$, $Z_\star$ e $Z_{neb}$);
  \item fração de população estelar jovem ($t_\star < t_{SF}$) maior que 5\% ($x_Y > 5$\%);
  \item $\tauV$ e $\tauVN$ maiores que 0.05;
  \item mais do que cinco zonas contribuindo para o cálculo dos perfis radiais (ver
  \ref{sec:synvsneb:amostra:rad});
  \item distância maior que 70\% do raio que contém metade da luz ({\em half-light radius} - HLR).
\end{itemize}
\noindent O que aqui chamamos de população jovem discutiremos um pouco mais adiante, na Sec.
\ref{sec:synvsneb:SFR}. última imposição é feita para que não haja contaminação por zonas
do bojo da galáxia (partes centrais onde as linhas são produzidas por diferentes fenômenos físicos,
relacionados a um núcleo ativo). Esse valor (0.7 HLR) foi definido por nossos colaboradores
analisando as curvas de brilho das galáxias e representa um valor máximo para localização de zonas
centrais. \ojo Ao final, produzimos a Fig. \ref{fig:amostraRealMorf} da mesma forma que a
Fig. \ref{fig:amostraMorf}, mas com a densidade superficial de massa ao invés da massa, pelo fato de
que as zonas não possuem mesma área. Nos sobram 16840 zonas de 199 galáxias (21 Sa, 42 Sb, 63 Sbc,
59 Sc e 14 Sd). Podemos ver que as tendências observadas na Fig. \ref{fig:amostraMorf} não se
alteram após estes cortes.

\begin{figure}
	\centering
	%\resizebox{0.99\textwidth}{!}{\includegraphics{Nh_logt_metBase_Padova2000_salp.pdf}}
	\resizebox{0.99\textwidth}{!}{\includegraphics{figuras/sample_real_morf_zones.pdf}}
	\caption[Classificação por morfologia após máscara.]
	{Mesma ideia da Fig \ref{fig:amostraMorf} no entanto para todas as zonas restantes após a
	aplicação da máscara discutida em \ref{sec:synvsneb:amostra:mask}. No painel superior esquerdo
	temos a densidade superficial de massa, diferentemente da Fig. \ref{fig:amostraMorf} pois as zonas
	possuem números diferentes de píxeis, portanto, áreas diferentes. Também podemos ver acima deste
	painel o número de zonas contidas em cada classe morfológica. Vemos que as tendências notáveis para
	os valores integrados se mantém quando analisamos os resultados por zona.}
	\label{fig:amostraRealMorf}
\end{figure}

\subsection{Perfis radiais}
\label{sec:synvsneb:amostra:rad}

Uma maneira interessante de analisar galáxias é produzir perfis radiais para as propriedades
físicas. Esse tipo de média azimutal (tanto em classes definidas por anéis circulares como por
anéis elípticos) diminui o espalhamento dos pontos e também permite estudo da evolução das
propriedades ao longo do raio da galáxia. Para que seja possível um ``empilhamento'' de galáxias,
estas médias radiais são feitas em um raio efetivo para cada galáxia. No nosso caso utilizamos como raio
efetivo aquele que comporta metade da luz da galáxia ({\em half-light radius} - HLR) e definimos 30
aneis com espessura de 0.1 HLR partindo do pixel central de cada galáxia. No artigo de
\citet{GonzalezDelgado.etal.2014a} os autores discutem as estruturas radiais de algumas propriedades
estelares, aplicando este tipo de estudo para 107 galáxias contidas no CALIFA {\em Survey}. Nele são
derivados os raios de metade da luz (HLR) e de metade da massa ({\em half-mass radius} - HMR) e
mostram que o valor de idade estelar média, extinção estelar e densidade superficial de massa
estelar são bem representados pelo seus valores medidas a 1 HLR.

O empilhamento é um processo aonde podemos comparar os perfis radiais de uma galáxia com os de outra
para assim poder criar perfis radiais de um grupo de galáxias que pertencem a determinada classe.
Por exemplo, pode-se fazer o perfil radial médio de um grupo de galáxias com a mesma classificação
morfológica e depois comparar esse perfil médio para cada tipo morfológico e assim por diante.
Dentro de nosso trabalho utilizamos os resultados para perfis radiais e também para zonas, nos
possibilitando portanto verificar diferenças nestes tipos de abordagens diferentes.

Apenas como exemplo, podemos observar na Fig. \ref{fig:K0140xYRadProf} o perfil radial da fração de
luz proveniente das populações jovens para a galáxia NGC1667 (objeto CALIFA 140). Nos dois primeiros
painéis temos a imagem do \SDSS do mesmo objeto e o mapa de $x_Y$ calculado pelo \starlight. No
últimos vemos o perfil radial juntamente com o valor de $x_Y$ para todas as zonas presentes na
galáxia e que compõe nossa amostra.

\begin{figure}
	\centering
	%\resizebox{0.99\textwidth}{!}{\includegraphics{Nh_logt_metBase_Padova2000_salp.pdf}}
	\resizebox{0.99\textwidth}{!}{\includegraphics{figuras/K0140_xY_radialProfile_realsample.pdf}}
	\caption[Imagem e perfil radial de $x_Y$.]
	{\emph{Painel esquerdo}: Imagem da galáxia NGC1667 (CALIFA 140) do \SDSS. \emph{Painel central}:
	Mapa da fração percentual de luz proveniente das populações jovens ($x_Y$). \emph{Painel direito}:
	$x_Y$ contra a distância radial. Os pontos em cinza ao fundo são os valores para zonas. O perfil radial
	calculado em bins elipticos com espessura de 0.1 HLR é desenhado com uma linha preta contínua. Já
	linha tracejada representa o valor da mediana de $x_Y$ em bins de distância radial. Vemos que o
	valor do perfil radial de $x_Y$ acompanha o valor da mediana. }
	\label{fig:amostraRealMorf}
\end{figure}
 
% Figuras:
% - histograma tipos - histograma massa - influências dos cortes em tauV, raio e x_Y - Anexo: Lista
% de galáxias com massa, redshift, idade, etc ...

\section{Comparação entre as taxas de formação estelar}
\label{sec:synvsneb:SFR}

Com a síntese de populações estelares podemos calcular a história de formação estelar utilizando o
vetor cumulativo de massa ($\eta_\star(t_\star)$), que abarca a fração total de massa convertida em
estrelas para cada idade das populações da base ($\mu_j$). Então podemos calcular uma taxa de
formação dentro de um intervalo de tempo:
\begin{eqnarray}
	\eta_\star(t_\star)\ &=&\ \sum\limits_{t_{\star,j} < t_\star} \mu_j \\
	\overline{\mathrm{SFR}_\star}(t_\star)\ &=&\ M_\star \frac{(1\ -\ \eta_\star(t_\star))}{t_\star},
\end{eqnarray}
\noindent onde M${}_\star$ é a massa total convertida em estrelas durante toda a história de
formação estelar de uma galáxia. 

Como explicado em \ref{sec:emlines:SFR}, podemos medir a taxa de formação estelar recente medindo a
luminosidade intrínseca de \Halpha. Com as duas taxas de formação estelar, uma recente e uma em
função do tempo ($t_\star$) podemos encontrar o tempo que melhor correlaciona as duas medidas, de
forma a encontrar uma escala de tempo que defina as populações jovens ($t_{SF}$), ou seja,
populações recém formadas e que geralmente ainda residem nas regiões de formação estelar. Na Fig.
\ref{fig:SFRsynvsneb} vemos a correlação entre as SFRs, calculando
$\overline{\mathrm{SFR}_\star}(t_\star)$ para diferentes idades. Neste gráfico os cortes
definidos em \ref{sec:synvsneb:amostra} com relação a idade e coeficiente de extinção mínimos não
estão aplicados. O mesmo é válido para cortes em distância radial. Encontramos $t_{SF}$ como sendo
próximo a 32 milhões de anos. Esse número não está muito distante da escala de tempo de vida das
estrelas que produzem a maioria dos fótons capazes de produzirem a linha de \Halpha ($\sim10^7$
anos). Vemos também que o valor máximo de correlação não altera substancialmente entre $10^7$ e
$10^{7.5}$.

\begin{figure}
	\centering
	%\resizebox{0.99\textwidth}{!}{\includegraphics{Nh_logt_metBase_Padova2000_salp.pdf}}
	\includegraphics[scale=0.7, clip]{figuras/Rs_allSFR.pdf}
	\caption[Comparação entre as SFR.]
	{\emph{Painel superior}: O coeficiente de correlação de Spearmann entre as SFR para diferentes
$t_\star$. Para cada cor temos um tipo de cálculo de SFR e em linha preta contínua temos os valores
do coeficiente de Spearmann para os perfis radiais de $\Sigma_{\mathrm{SFR}}$, que possui o valor de
idade que utilizamos para ser nossa escala de tempo de formação estelar ($t_{SF})$. \emph{Demais
painéis}: Cada um dos gráficos de comparação entre as SFR utilizando $t_{SF}$. Essa idade foi a
escolhida pois é a que possui o melhor coeficiente de correlação entre os perfis radiais da
densidade de coluna da SFR $\Sigma_{\mathrm{SFR}}$. As correlações entre densidades de coluna são
mais confiáveis pois removem o termo $d^2$ existente no cálculo da SFR que induz uma correlação
direta entre $\mathrm{SFR}_\star$ e $\mathrm{SFR}_{\Halpha}$. Em cada painel a linha pontilhada
vermelha é o ajuste linear utlizando OLS bisector, em azul é o ajuste forçando que a inclinação
seja 1 e em preto é a bissetriz ($x = y$).}
	\label{fig:SFRsynvsneb}
\end{figure}
% Figuras:
% - melhor comparação e ajuste

\section{Comparação entre os coeficientes de extinção}
\label{sec:synvsneb:tauv}

A síntese de populações estelares realizadas pelo \starlight adota o mesmo modelo de extinção por
poeira explicado em \ref{sec:emline:datacube:tauvneb}, onde todas as populações são atenuadas pelo
mesmo fator $e^\tau_\lambda$. Essa simplificação contraria tanto evidências observacionais quanto
estudos teóricos, que caminham para um cenário onde populações mais jovens são mais atenuadas pela
poeira que populações mais velhas. 

Apesar do modelo de extinção ser o mesmo, o coeficiente calculado por cada um dos procedimentos é
diferente, como podemos ver na Fig. \ref{fig:tauVsynvsneb}. Esta figura apresenta a comparação entre
os coeficientes de extinção para zonas e também para os perfis radiais. O coeficiente que vem do
\starlight é calculado no processo de ajuste espectral de maneira a evitar degenerescências no
processo, já o do decremento balmer representa melhor as regiões onde existem os observáveis
necessários para seu cálculo, ou seja, regiões onde existam linhas de \Halpha e \Hbeta, portanto,
regiões mais jovens. Vamos nos aprofundar um pouco melhor nessa diferença entre os coeficientes de
extinção no próximo capítulo.

\begin{figure}
	\centering
	%\resizebox{0.99\textwidth}{!}{\includegraphics{Nh_logt_metBase_Padova2000_salp.pdf}}
	\resizebox{0.99\textwidth}{!}{\includegraphics{figuras/CompareTauV.pdf}}
	\caption[Comparação entre os coeficientes de extinção.] 
	{Comparação entre os coeficientes de extinção por poeira provenientes da síntese ($\tauVS$) e
	do decremento de Balmer ($\tauVN$). Os contornos azul, amarelo e vermelho representam os intervalos
	de confiança ($1\sigma$, $2\sigma$ e $3\sigma$). A linha preta representa a mediana e as linhas
	pontilhadas representam o ajuste utilizando OLS bisector (vermelha) e mínimos quadrados (preta).}
	\label{fig:tauVsynvsneb}
\end{figure}

% Figuras:
% - Exemplo de diferenças entre mapas de tau_V
% - comparação entre tauV
% - x_Y

\section{Comparação entre metalicidade}
\label{sec:synvsneb:Z}

No artigo de \citet{GonzalezDelgado.etal.2014b}, (GD14 daqui em diante) cuja versão completa está no
apêndice \ref{apendice:1} e eu tenho participação, analisamos a relação entra massa estelar ($M_\star$)
e metalicidade estelar ($Z_\star$) de uma amostra de 300 galáxias do CALIFA. Vamos que estas seguem
uma boa relação massa metalicidade (MZR) como em \citet{Tremonti.etal.2004a}, que analisa a relação
embora para a metalicidade do gás ($Z_{neb}$). A relação totalmente estelar é mais inclinada que a
relação comparando com a metalicdade do gás pois o intervalo de metalicidades estelares é maior. A
metalicidade estelar é calculada segundo a equação:
\begin{equation}
 	\label{eq:logZmass}
 	\langle \log Z_\star \rangle_{M,xy} = 
	\frac{ \sum_{tZ} M_{\star,tZ,xy} \times \log\ Z}{
	\sum_{tZ} M_{\star,tZ,xy} }.
\end{equation}
Comparando nossos resultados com os obtidos por \citet{Sanchez.etal.2013a} (Fig. 2b em GD14) para
MZR em regiões \Hii vemos que eles se distanciam conforme $M_\star$ diminui, mas ao calcularmos a
metalicidade para populações mais jovens que 2 bilhões de anos, os resultados se aproximam
bastante. 
%Quando calculamos a metalicidade média das populações jovens das galáxias, estamos olhando
para as populações estelares de galáxias menos massivas, mais jovens e com atividade de formação estelar
intensa, portanto com um interva

% Figuras:
% - comparação entre metalicidades

%% End of this chapter
