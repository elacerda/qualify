%%%%%%%%%%%%%%%%%%%%%%%%%%%%%%%%%%%%%%%%%%%%%%%%%%%%%%%%%%%%%%%%%
% Qualificacao de Doutorado / Dept Fisica, CFM, UFSC            %
% Eduardo@UFSC - 2015                                           %
%%%%%%%%%%%%%%%%%%%%%%%%%%%%%%%%%%%%%%%%%%%%%%%%%%%%%%%%%%%%%%%%%

%:::::::::::::::::::::::::::::::::::::::::::::::::::::::::::::::%
%                                                               %
%                          Capítulo 3                           %
%                                                               %
%:::::::::::::::::::::::::::::::::::::::::::::::::::::::::::::::%

%***************************************************************%
%                                                               %
%                     Comparação SFR EML                        %
%                                                               %
%***************************************************************%

\chapter{Propriedades da síntese e propriedades nebulares}
\label{sec:synvsneb}

\section{Amostra de galáxias}
\label{sec:synvsneb:amostra}

Neste trabalho estamos interessados em estudar a formação estelar recente em discos de galáxias.
Nossa amostra contém 293 galáxias com tipos morfológicos entre Sa e Sd, conforme a Fig.
\ref{fig:amostraMorf}. Essas galáxias possuem massas entre \ojo xxx e xxx e idades entre xxx e xxx
\myojo{blue}{conforme Fig xxx ?}. Escolhemos também uma amostra de galáxias com relação entre os
semieixos maior que 0.7 ($\frac{b}{a} > 0.7$). Estas galáxias também aparecem na Fig.
\ref{fig:amostraMorf}.

Destas galáxias, utilizamos apenas regiões que possuam ao menos 5\% de população jovem. O que
aqui chamamos de população jovem discutiremos um pouco mais adiante, na Sec. \ref{sec:synvsneb:SFR}.
Para uma região (zona) de uma galáxia ser considerada por nosso estudo ela precisa ter:
\begin{itemize}
  \item medidas do fluxo integrado das linhas de \Halpha, \Hbeta, \nII e \oIII com relação
 sinal-ruído maior do que 3;
  \item medidas de $\tau_V^\star$ e $\Sigma_{SFR}^\star$;
  \item fração de população estelar jovem ($t_\star < t_{SF}$) maior que 5\% ($x_Y > 5$\%);
  \item $\tau_V^\star$ e $\tau_V^{neb}$ maiores que 0.05;
  \item distância maior que 0.7 raio de meia luz ({\em half-light radius} - HLR).
\end{itemize}
Esta última imposição é feita para que não haja contaminação por zonas do bojo da galáxia
(partes centrais onde as linhas são produzidas por diferentes fenômenos físicos, relacionados a um
núcleo ativo). Esse valor (0.7 HLR) foi encontrado por nossos colaboradores e representa um valor
máximo para localização de zonas centrais. Atualmente não há nenhuma publicação em revistas
científicas, mas um de nossos colaboradores acaba de defender sua tese de doutorado estudando a
decomposição bojo-disco usando as galáxias do CALIFA \myojo{red}{REF.}. Logo teremos atualizações
nesta área. \ojo Ao final, nos sobram xxx zonas de xxx galáxias.
% Figuras:
% - histograma tipos - histograma massa - influências dos cortes em tauV, raio e x_Y - Anexo: Lista
% de galáxias com massa, redshift, idade, etc ...

\section{Comparação entre as taxas de formação estelar}
\label{sec:synvsneb:SFR}

Com a síntese de populações estelares podemos calcular a história de formação estelar utilizando o
vetor cumulativo de massa ($\eta_\star(t_\star)$), que abarca a fração total de massa convertida em
estrelas para cada idade das populações da base ($\mu_j$). Então podemos calcular uma taxa de
formação dentro de um intervalo de tempo:
\begin{eqnarray}
	\eta_\star(t_\star)\ &=&\ \sum\limits_{t_{\star,j} < t_\star} \mu_j \\
	\overline{\mathrm{SFR}_\star}(t_\star)\ &=&\ M_\star \frac{(1\ -\ \eta_\star(t_\star))}{t_\star},
\end{eqnarray}
\noindent onde M${}_\star$ é a massa total convertida em estrelas durante toda a história de
formação estelar de uma galáxia. 
Como explicado em \ref{sec:emlines:SFR}, podemos medir a taxa de formação estelar recente medindo a
luminosidade intrínsica de \Halpha. Com as duas taxas de formação estelar, uma recente e uma em
função do tempo (t_\star) podemos encontrar o tempo que melhor correlaciona as duas medidas, de
forma a encontrar uma escala de tempo que defina as populações jovens, ou seja, populações recém
formadas e que geralmente ainda residem nas regiões de formação estelar. Na Fig.
\ref{fig:SFRsynvsneb} vemos a correlação entre as SFRs, calculando
$\overline{\mathrm{SFR}_\star}(t_\star)$ para diferentes idades.

% Figuras:
% - comparação SFR diferentes timescales
% - melhor comparação e ajuste

\section{Comparação entre os coeficientes de extinção}
\label{sec:synvsneb:tauv}
% Figuras:
% - Exemplo de diferenças entre mapas de tau_V
% - comparação entre tauV
% - x_Y

\section{Comparação entre metalicidade}
\label{sec:synvsneb:Z}
% Figuras:
% - comparação entre metalicidades

%% End of this chapter
