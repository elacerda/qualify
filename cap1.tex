%%%%%%%%%%%%%%%%%%%%%%%%%%%%%%%%%%%%%%%%%%%%%%%%%%%%%%%%%%%%%%%%%
% Qualificacao de Doutorado / Dept Fisica, CFM, UFSC            %
% Eduardo@UFSC - 2015                                           %
%%%%%%%%%%%%%%%%%%%%%%%%%%%%%%%%%%%%%%%%%%%%%%%%%%%%%%%%%%%%%%%%%

%:::::::::::::::::::::::::::::::::::::::::::::::::::::::::::::::%
%                                                               %
%                          Capítulo 1                           %
%                                                               %
%:::::::::::::::::::::::::::::::::::::::::::::::::::::::::::::::%

%***************************************************************%
%                                                               %
%                         Introdução                            %
%                                                               %
%***************************************************************%

\chapter{Introdução}
\label{sec:intro}

%***************************************************************%
%                                                               %
%                         Intro - IFS                           %
%                                                               %
%***************************************************************%

\section{Espectroscopia de campo integral}

Na última década presenciamos uma proliferação de surveys de imageamento e
espectroscopia. Surveys como o SDSS \citep{York2000}, ALHAMBRA \citep{Moles2008}
e COSMOS \citep{Scoville2007}, para citar alguns exemplos, permitem explorar a
distribuição espectral de energia (SED\footnote{\em Spectral Energy
Distribution.}) de centenas de milhares a milhões de galáxias.
Entretanto, da forma como estes surveys são executados, há sempre um compromisso
entre a resolução espacial e a espectral. As imagens obtidas pelo SDSS têm um
boa resolução espacial, mas mapeiam a SED de forma grosseira, com apenas $5$
filtros de banda larga ($ugriz$). Já os espectros obtidos pelo mesmo survey
possuem uma excelente resolução e cobertura espectral, mas obtém o espectro
integrado da região central das galáxias.

O melhor dos dois mundos pode ser alcançado com instrumentos que fazem
espectroscopia de campo integral (IFS\footnote{\em Integral Field
Spectroscopy.}). Instrumentos que realizam este tipo de espectroscopia consistem
em geral de um amontoado de fibras óticas, as quais alimentam um espectrógrafo
comum. Assim, depois de um processo relativamente complicado de redução de
dados, obtém-se espectros espacialmente resolvidos com uma boa resolução
espectral e espacial. O survey CALIFA ({\em Calar Alto Legacy Integral Field
Area survey\footnote{\url{http://www.caha.es/CALIFA/}}}) está utilizando o
instrumento PMAS/PPAK do observatório Calar Alto para obter IFS de $600$
galáxias \citep{Sanchez2012}.
Destas, $100$ já estão disponíveis no primeiro {\em Data Release}
\citep[DR1]{Husemann2013}, afirmando o caráter de legado deste survey.

Quando completado, o CALIFA terá obtido da ordem de $10^6$ espectros, quase o
mesmo que o SDSS. Porém, este não será apenas mais um survey espectroscópico. A
riqueza do CALIFA está nas informações espacialmente resolvidas, em uma amostra
representativa do universo local ($0.005 < z < 0.03$, limitada em diâmetro
angular) cobrindo a distribuição de galáxias no diagrama cor--magnitude da nuvem
azul à sequência vermelha, amostrando galáxias todos os tipos morfológicos
(espirais, elípticas, irregulares e até mesmo alguns sistemas em interação).


%***************************************************************%
%                                                               %
%                Intro - Stellar synthesis of IFS               %
%                                                               %
%***************************************************************%

\section{Síntese de população estelar espacialmente resolvida}
\label{sec:Intro:Sintese}

Os espectros espacialmente resolvidos do CALIFA podem ser descritos como um cubo
de dados, com as duas primeiras dimensões sendo a posição $x$ e $y$ (ascensão
reta e declinação) e a terceira sendo o comprimento de onda. Nestes cubos,
planos com comprimento de onda constante são imagens, enquanto ``fatias''
definidas por um par $(x, y)$ constante são espectros. Pode-se tratar estes
espectros individualmente, embora na verdade, nos cubos do CALIFA os pixels
vizinhos estão correlacionados devido ao {\em seeing} do céu e ao processo de
observação. Há a queda na relação sinal/ruído (S/N) nas regiões mais
afastadas do núcleo da galáxia, onde o brilho superficial é muito menor.
Algumas galáxias possuem outros objetos ``intrusos'' que precisam ser
mascarados. Linhas telúricas\footnote{Linhas de absorção causadas pela
atmosfera.} também precisam ser mascaradas. Assim, em geral, é necessário um
preprocessamento visando manter um S/N mínimo e garantir um espectro livre de
contaminação. Para mais detalhes sobre o preprocessamento utilizado neste
trabalho, ler a seção 3 do Apêndice \ref{apendice:PaperResolving1}.

% \TODO: o que nós fazemos?

Um aspecto importante do preprocessamento utilizado é que o cubo de dados é
dividido em zonas de Voronoi, onde regiões com baixo S/N são combinadas formando
efetivamente ``pixels maiores''. Desta forma, o cubo original é transformado
numa matriz de zonas e comprimento de onda, onde fatias de zona constante são
espectros. Com isso, os espectros, e as máscaras e erros que os acompanham,
estão prontos para serem usados pelo próximo passo.

A síntese de população estelar consiste em obter a história de formação estelar
(SFH) de uma galáxia utilizando modelos de população estelar simples (SSP).
Ajusta-se o espectro de uma galáxia como a soma de espectros de SSPs com idades
e composições químicas distintas (levando em conta a atenuação por poeira). O
resultado é um vetor de frações de luz e massa destas SSPs, que podem ser
facilmente convertidos a uma SFH conforme a prescrição de \cite{Asari2007}.
O programa utilizado é o \starlight, desenvolvido por \cite{CidFernandes2005}.

Considerando cada zona como uma galáxia distinta\footnote{O que é razoável, já
que cada pixel corresponde em geral a uma distância de centenas de parsecs,
comportando facilmente vários aglomerados e regiões de formação estelar.},
alimentamos todos os espectros das zonas ao \starlight, obtendo o resultado da
síntese como um arquivo de síntese separado para cada zona. Entretanto, para
visualizar ou mesmo tentar entender estes resultados, é preciso organizar e
converter estes arquivos para um formato mais adequado.


%***************************************************************%
%                                                               %
%                        Intro - PyCASSO                        %
%                                                               %
%***************************************************************%

\section{O nascimento do PyCASSO}

Todo o descrito anteriormente forma o alicerce deste presente trabalho. Da
necessidade de manipular os resultados da síntese dos cubos de dados das
galáxias do CALIFA, nasceu o software PyCASSO.

PyCASSO ({\em Python CALIFA \starlight Synthesis Organizer}) é um software
desenvolvido em Python com o objetivo de gerenciar os dados produzidos pelo
\starlight com base nos dados do CALIFA. O Capítulo \ref{sec:pycasso} apresenta
a documentação do PyCASSO, e em seguida discute os artigos publicados que o
utilizam intensivamente.


%***************************************************************%
%                                                               %
%                      Intro - Decomposicao                     %
%                                                               %
%***************************************************************%

\section{Decomposição morfológica de galáxias}

Como aplicação científica do PyCASSO, o Capítulo \ref{sec:Decomp} trata da
síntese espectral das componentes morfológicas\footnote{As componentes
morfológicas de uma galáxia, tratadas neste trabalho, são o bojo e o disco.} de
uma galáxia. As componentes, são determinadas através de ajuste de perfil de
brilho, para cada comprimento de onda. Assim se obtém um cubo de espectros para
cada componente morfológica.Estes cubos são tratados como galáxias separadas, e
seus espectros são passados pelo \starlight. O resultado são cubos extras de
síntese espectral, os quais podem ser comparados com a síntese da galáxia
original. Este estudo está em progresso no momento, os resultados aqui
apresentados são preliminares.

% End of this chapter
