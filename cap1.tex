%%%%%%%%%%%%%%%%%%%%%%%%%%%%%%%%%%%%%%%%%%%%%%%%%%%%%%%%%%%%%%%%%
% Qualificacao de Doutorado / Dept Fisica, CFM, UFSC            %
% Eduardo@UFSC - 2015                                           %
%%%%%%%%%%%%%%%%%%%%%%%%%%%%%%%%%%%%%%%%%%%%%%%%%%%%%%%%%%%%%%%%%

%:::::::::::::::::::::::::::::::::::::::::::::::::::::::::::::::%
%                                                               %
%                          Capítulo 1                           %
%                                                               %
%:::::::::::::::::::::::::::::::::::::::::::::::::::::::::::::::%

%***************************************************************%
%                                                               %
%                         Introdução                            %
%                                                               %
%***************************************************************%

\chapter{Introdução}
\label{sec:intro}

Com a criação e difusão em massa de dados gerados com os grandes projetos astronômicos atuais, nossa
forma de enxergar o mundo vem se tornando cada vez mais acurada quanto ao Universo. Com diferentes
faixas espectrais (desde raios-$\gamma$ até micro-ondas), diferentes fontes de dados (espectros de
galáxias integradas, espectroscopia de campo, imagens, monitoramento temporal) e diferentes
objetivos, os {\em surveys} astronômicos\footnote{Um survey astronômico é um levantamento de
informações ou mapeamento de regiões do céu utilizando telescópios e detetores.} permeiam por
diferentes fenônmenos astrofísicos. Além de estarem formando um imenso legado de informações para
futuros astrofísicos, são basilares para o desenvolvimento de novas ideias e resolução dos desafios
atuais da área.
 
\section{O GAS-UFSC e o IAA-CSIC}
\label{sec:intro:UFSCeIAA}

Nos últimos anos nosso grupo de Astrofísica (GAS-UFSC) aqui na Universidade Federal de Santa
Catarina vem trabalhando com dados de diversos {\em surveys}. Nosso grupo foi pioneiro no estudo das
propriedades físicas das populações estelares de aproximadamente um milhão de galáxias do \SDSS
através do projeto
SEAGal/\starlight\footnote{\href{http://starlight.ufsc.br}{http://starlight.ufsc.br}} publicando
diversos artigos importantes e amplamente citados \citep[e.g., ][]{CidFernandes.etal.2005a,
Mateus.etal.2006a, Stasinska.etal.2006a, Asari.etal.2007a, Stasinska.etal.2008a,
CidFernandes.etal.2011a}.

Atualmente participamos de um projeto entre nosso grupo de populações estelares aqui no GAS com
pesquisadores do Instituto de Astrofísica de Andalucía (IAA), na cidade de Granada, Comunidade
autônoma de Andalucía, ao sul da Espanha. Esse instituto pertence ao {\em Consejo Superior de
Investigaciones Científicas} (CSIC), o maior órgão público (estatal) de pesquisas científicas na
Espanha, e o terceiro maior da Europa. Conta com pesquisadores participantes do CALIFA
\citep[][]{Sanchez.etal.2012a}, funcionando como centro físico do projeto. A pesquisadora Rosa M.
González Delgado, coorientadora deste trabalho, uma das principais líderes do projeto e que também
atua como Pesquisadora Visitante Especial (PVE-CsF) aqui na UFSC; Rubén García Benito, que faz parte
do grupo de redução dos dados do {\em survey}; e Enrique Pérez, do grupo de populações estelares, já
trabalham em nossa parceria e possuem conhecimento e domínio das técnicas exploradas por nosso
projeto, além de participarem ativamente do desenvolvimento do CALIFA. Durante os últimos três anos
nosso grupo de populações estelares no CALIFA publicou diversos artigos \citep[e.g.,
][]{Perez.etal.2013a, GonzalezDelgado.etal.2014a, GonzalezDelgado.etal.2014b,
GonzalezDelgado.etal.2015a}, e mais dois já foram submetidos. Paralelamente participamos de diversos
congressos e conferências publicando nossos resultados.

%***************************************************************%
%                                                               %
%                         Intro - f_gas                         %
%                                                               %
%***************************************************************%

\section{Gás, estrelas e poeira nas galáxias}
\label{sec:intro:galaxias}

Galáxias são formadas por uma complexa mistura de gás, poeira, estrelas e matéria escura,
distribuídas em discos, bulbos e halos. O gás é o combustível da formação estelar. As nuvens de gás
molecular, formadas pelo esfriamento de gás do meio interestelar, se fragmentam formando estruturas
menores e cada vez mais densas, que são chamadas {\em clumps}. A formação estelar acontece quando o
centro dessas massas de gás colapsam devido ao desbalanceamento entre pressão e gravidade. Essas
regiões, que podem ser pequenas ou se estenderem a gigantes berçários estelares, estão geralmente
cobertas por uma densa camada de poeira. Gerada pelo próprio processo de formação estelar, a poeira
funciona como uma cortina que modifica a energia dos fótons que chegam até nossos detectores. Isso
acontece pois os grãos de poeira absorvem e reemitem radiação com diferentes energias, modificando o
espectro observado. Esse processo é chamado de extinção por poeira. Apesar de modificar o espectro,
a poeira pode também ser usada como sinalizadora de regiões aonde há intensa formação estelar. No
final do ciclo de vida das estrelas, diversos elementos são jogados no meio interestelar através das
explosões de supernovas, alterando assim a composição química do gás disponível para produção de
novas estrelas.

\citet{Schmidt.1959a} foi o primeiro a propor a existência de uma lei de potências que liga a taxa
de formação estelar (SFR - {\em star formation rate}) e o gás. Anos depois, \citet{Kennicutt.1998a}
estuda essa relação observacionalmente, utilizando diversos indicadores de formação estelar em
diferentes faixas espectrais.
Em seu trabalho, Kennicutt estabelece a ligação entre a densidade superficial do gás e da SFR. Hoje
em dia essa é comumente chamada de relação de Kennicut-Schmidt (KS) ou lei de formação estelar. A
equação parametrizada por \citeauthor{Kennicutt.1998a} foi:
\begin{equation}
	\Sigma_{\mathrm{SFR}}\ =\ (2.5\pm0.7)\times 10^{-4} \left(\frac{\Sigma_{\mathrm{gas}}}{
M_\odot\ \mathrm{pc}^{-2}}\right)^{1.4 \pm 0.15}\ M_\odot\ \mathrm{yr}^{-1}\ \mathrm{kpc}^{-2}.
	\label{eq:SFRKennicutt}
\end{equation}

Utilizando a síntese de populações estelares aplicada às galáxias do CALIFA poderíamos estimar
$\SigmaGas$ através de uma equação como \eqref{eq:SFRKennicutt} mas, além de não ser uma maneira
independente de $\SigmaSFR$, esta equação foi parametrizada para valores integrados de galáxias,
diferentemente da área típica das regiões que estamos observando (regiões de galáxias com tamanhos
típicos de $\sim 1 kpc$). Com a síntese é possível também obter a história de formação estelar
através da fração de populações estelares com distintas idades \citep{Asari.etal.2007a}, não necessitando assim
prender-se às zonas das galáxias onde o espectro tenha relação sinal ruído (S/N) suficiente para a
medida de todas as linhas espectrais necessárias para os cálculos sobre o gás e poeira. Nos
falta, porém, uma medida da quantidade de gás de forma independente da formação estelar, o que nos
levou a buscar uma conversão que utilize uma ou mais propriedades provenientes da síntese com o
\starlight.
% Os mapas de regiões formadoras de estrelas juntamente com os espectros residuais\footnote{Um
% espectro residual é resultados da subtração entre o espectro observado e o modelado através da
% síntese de populações estelares.} contém informações sobr e a poeira dessas distintas regiões

\section{De poeira para gás}
\label{sec:intro:dust2gas}

Dentro da astrofísica extragaláctica, um dos grandes temas ainda incômodo é a determinação de massas
de gás. A quantidade de gás de uma galáxia define o processo de formação estelar, portanto, é peça
fundamental no entendimento da sua evolução. O gás atômico (formado majoritariamente por átomos de
hidrogênio) geralmente está nas partes de fora das regiões de formação estelar, aonde a temperatura
já está alta o suficiente para quebrar a ligação entre os hidrogênios. O gás molecular é associado
diretamente com a formação estelar, porém a molécula de $\mathrm{H}_2$ não possui um observável
direto. A baixa massa da molécula não permite que nenhum grau de liberdade rotacional seja excitado,
pois a temperatura nessas regiões é muito baixa, necessitando ser traçado através de outros
elementos presentes ali. Geralmente o que se usa são as transições entre os níveis energéticos do
monóxido de carbono (CO) pois suas linhas são bem fortes (fácil de observar).
\citet{Bolatto.etal.2013a} revisam o tema mostrando um panorama geral sobre a conversão de
intensidade da linha de CO e a densidade superficial do gás.

Avermelhamento (medido através de índices de cores) e extinção da luz das estrelas são temas
discutidos e utilizados como evidência da existência de {\em ``núvens interestelares''}.
\citet{Barnard.1908a}, através da análise de imagens dos objetos Messier 8, 17 e 18, discute a
presença de ``buracos ou manchas'' entre as estrelas, e a existência de estrelas no meio de
``nebulosidades''. Esse artigo foi escrito inclusive antes da famosa discussão de Sharpley e Curtis
em 1920, e do artigo de \citet{Hubble.1925a} que definiu que as ``nebulosas espirais'' eram objetos
que estavam a distâncias muito maiores que a dimensão da Via Láctea. Atualmente, trabalhos como a
revisão de \citet{Draine.2003a} nos ajudam a entender as propriedades físicas, a natureza dos
grãos de poeira e de como ela afeta a luz que a atravessa.

Uma das formas de se mapear e determinar quantidades de gás em galáxias é utilizando a presença de
poeira. Devido a presença de poeira nas regiões de formação estelar, podemos utilizá-la como maneira
indireta de estimar a densidade superficial do gás oculto através da introdução de um fator de
conversão:
\begin{equation}
	\label{eq:dust2gas}
	\SigmaGas\ =\ \frac{\SigmaGas}{\SigmaDust} \times \SigmaDust\ = \kappa \times \SigmaDust. 
\end{equation}
\noindent O índice d designa poeira ({\em dust}, em inglês) e as densidades superficiais são dadas
em $M_\odot pc^{-2}$. Diversos artigos discutem a calibração e os {\em caveats} deste tipo de
conversão e também problemas de degenerescência na conversão CO-$\mathrm{H}_2$ \citep{Guiderdoni.Rocca.1987,
Leroy.etal.2011a, Leroy.etal.2013a, RemyRuyer.etal.2014a}, mas quase todos eles utilizam medidas de
infravermelho (IR) para estimar $\Sigma_{\mathrm{d}}$, onde os grãos de poeira geralmente
re-emitem a luz absorvida em outras frequências. Já \citet[][BR13 daqui em
diante]{Brinchmann.etal.2013a} desenvolvem uma conversão de poeira para gás utilizando medidas da
poeira em absorção, e não em emissão, como os estudos em IR, para as galáxias formadoras de
estrelas\footnote{Galáxias {\em star-forming} (SF) - com formação estelar ativa.} do sétimo
lançamento público de dados ({\em data release}) do \SDSS \citep[][DR7]{Abazajian.etal.2009a}.
Segundo BR13, esse é um método é sensível ao gas total da galáxia (soma do gás molecular com o gás
atômico). Na busca por traçadores de gás, escolhemos a poeira em absorção (na faixa do espectro
óptico) para tal objetivo pois é uma quantidade à qual temos acesso, tanto via síntese espectral
como por meio de linhas de emissão.

\section{Este trabalho}
\label{sec:intro:estetrabalho}

Nessa primeira etapa do doutorado estamos procurando entender melhor a natureza das propriedades
físicas estelares e nebulares e suas diferenças nas galáxias do CALIFA. No Cap.\ \ref{sec:emlines}
descrevemos o processo de medidas das propriedades nebulares. A seleção da amostras, suas ressalvas
e exemplos de perfis radiais são discutidos no Cap.\ \ref{sec:amostra}. Comparações entre algumas das
propriedades nebulares e àquelas provenientes da síntese são discutidas no Cap.\ \ref{sec:synvsneb}.
De maneira e melhor entender a relação entre o coeficiente de extinção proveniente da síntese e do
decremento de Balmer e suas relações com a extinção do meio interestelar ({\em interstellar medium}
- ISM) e das núvens formadoras de estrelas ({\em birth-clouds} - BC) desenvolvemos uma discussão
mais pontualmente no Cap.\ \ref{sec:difextin}. Por fim, no Cap.\ \ref{sec:gasfrac}, desenvolvemos a
conversão de poeira em absorção para gás utilizando o método de BR13, juntamente com algumas
proposições para os próximos passos deste projeto. Permeando os capítulos, quando pertinente,
descrevo minha participação em artigos nestes últimos dois anos.

% End of this chapter
