%%%%%%%%%%%%%%%%%%%%%%%%%%%%%%%%%%%%%%%%%%%%%%%%%%%%%%%%%%%%%%%%%
% Qualificacao de Doutorado / Dept Fisica, CFM, UFSC            %
% Eduardo@UFSC - 2015                                           %
%%%%%%%%%%%%%%%%%%%%%%%%%%%%%%%%%%%%%%%%%%%%%%%%%%%%%%%%%%%%%%%%%


%:::::::::::::::::::::::::::::::::::::::::::::::::::::::::::::::%
%                                                               %
%                          Capítulo 4                           %
%                                                               %
%:::::::::::::::::::::::::::::::::::::::::::::::::::::::::::::::%

%***************************************************************%
%                                                               %
%                           Conclusao                           %
%                                                               %
%***************************************************************%

\chapter{Conclusões e perspectivas}
\label{sec:conclusao}

O survey CALIFA está produzindo cubos de dados de IFS de $600$ galáxias. Já
foram publicados $100$ destes cubos no primeiro {\em data release}, e mais de
$200$ foram observados, mas ainda estão em processo de controle de qualidade e
embargo da colaboração. Através do programa \starlight, foram obtidos diversos
parâmetros relacionado às populações estelares componentes de cada pixel,
formando cubos de dados adicionais ao observados. Neste trabalho, foi
desenvolvido um programa chamado PyCASSO para organizar e analisar estes cubos
de dados. Este programa é utilizado por cerca de $10$ pessoas que estudam
populações estelares na colaboração do CALIFA. Bastante atenção foi dada à
documentação (\ref{sec:pycasso:Pycasso}), que pode ser vista no Apêndice
\ref{apendice:manual}. Foram publicados $4$ artigos que se baseiam fortemente no
uso de PyCASSO para análise e gráficos, resumidos na Seção \ref{sec:pycasso:art}
e presentes como apêndices deste trabalho (Apêndices
\ref{apendice:PaperResolving1}, \ref{apendice:PaperResolving2},
\ref{apendice:InsideOut} e \ref{apendice:RadStruct}).

PyCASSO já foi usado com sucesso com dados de outros surveys como o PINGS
\citep{RosalesOrtega2010}, um precursor do CALIFA, porém com até $10$ vezes mais
espectros por galáxia. Outros surveys IFS certamente seguirão.
MaNGA\footnote{\url{http://www.sdss3.org/future/manga.php}} está sendo planejado
pela colaboração SDSS, e deverá obter IFS de $10000$ galáxias\footnote{Muito
embora tenha um número muito maior de galáxias do que o CALIFA, sua cobertura
espacial será menor. Assim, estes survey deverá ser complementar ao CALIFA.}.
Testes com dados preliminares foram feitos com PyCASSO, requerendo apenas
pequenas modificações nos arquivos para que funcionem normalmente. A tendência
agora é que PyCASSO se torne um programa modular, onde o usuário cria uma
descrição do arquivo de IFS de forma que o programa saiba como ler os dados.

Como seguimento ao programa de doutorado, começou-se a desenvolver uma técnica
de síntese espectral aplicada às componentes morfológicas (bojo e disco) de
galáxias. A decomposição morfológica do perfil de brilho em uma imagem de uma
galáxia é um campo de estudo bem desenvolvido, com ferramentas bastante
eficientes disponíveis na comunidade acadêmica (GALFIT, BUDDA e Imfit, por
exemplo). A ideia aqui foi se valer destas ferramentas e realizar a decomposição
para imagens em cada comprimento de onda dos cubos de espectro. Foram mostrados
resultados preliminares da decomposição.

O processo de decomposição ainda contém etapas manuais, e depende de uma boa
escolha inicial nos parâmetros para não cair em mínimos locais. Isto pode ser
resolvido utilizando outros algoritmos de minimização. Atualmente o algoritmo
utilizado é o de Levenberg-Marquadt. Imfit disponibiliza outros dois algoritmos,
menos sensíveis a mínimos locais, porém estes ainda ainda foram implementados no
código em Python.

Também é preciso estudar a forma como os parâmetros morfológicos deveriam variar
com o comprimento de onda. Do modo como o código está implementado, é possível
aplicar um filtro para suavizar os parâmetros após o ajuste, e repeti-lo com
alguns parâmetros fixos. Entretanto, não está claro quais parâmetros variam
suavemente (nem o quão suave e qual forma geral deveriam ter) e quais se espera
que mudem dramaticamente de um comprimento de onda a outro vizinho. Aqui
provavelmente a cinemática das populações estelares tenha um papel importante,
que poderá dar pistas sobre o comportamento das propriedades morfológicas.

Um estudo que já pode ser feito com os resultados atuais é medir a densidade
superficial de massa estelar do disco, que fica normalmente escondido atrás do
bojo. O objetivo é verificar previsão de \citet{Freeman1970}, que diz que a
densidade superficial de massa do núcleo dos discos de galáxias é sempre o
mesmo, independente do tipo morfológico e do perfil de massa que a galáxia
tenha. Quando o ajuste de bojo e disco estiverem dominados, o passo lógico
seguinte é tentar aplicar o ajuste a galáxias espirais. Os braços espirais têm
um papel importante na formação estelar, e com esta abordagem pode-se tentar
obter o histórico de formação estelar nestas regiões. Também é necessário
melhorar o ajuste para galáxias com um grande ângulo de inclinação. Com estas
mudanças, vai ser possível aplicar o estudo a uma boa fração da amostra do
CALIFA, e então avaliar estatisticamente a validade do método.


% End of this chapter
